\documentclass[a4paper,11pt,abstracton,hidelinks]{scrartcl}

\usepackage[margin=2cm]{geometry}
\usepackage{graphicx}
\usepackage[UKenglish]{babel}
\usepackage{csquotes}
\usepackage{float}
\usepackage[export]{adjustbox}
\usepackage[T1]{fontenc}
\usepackage{lmodern}
\usepackage[textsize=tiny]{todonotes}
\usepackage[labelsep=period,font=small,labelfont=bf,format=plain]{caption}
\captionsetup[table]{
  position=above,
  belowskip=10pt,
  aboveskip=0pt,
}
\usepackage[group-separator={,}]{siunitx}
\usepackage{booktabs}
\usepackage{pdflscape}
\usepackage{tablefootnote}
\usepackage{authblk}
\usepackage{threeparttable}
\usepackage{afterpage}
\usepackage{lineno}
\usepackage{setspace}
\usepackage{hyperref}
\usepackage{cleveref}
\crefformat{footnote}{#2\footnotemark[#1]#3}
\crefrangelabelformat{footnote}{#3#1#4--#5\footnotemark{#1}{#2}#6}

\usepackage{grbib}
\addbibresource{malvecbib/refs.bib}


\title{
The genomic landscape of recent positive selection in the African malaria vectors \textit{Anopheles gambiae} and \textit{Anopheles coluzzii}
}

\subtitle{\large{Supplemental material}}

\author{@@authors}

\begin{document}

\maketitle

\beginsupplement

\tableofcontents
\listoffigures
\listoftables

\clearpage


%%%%%%%%%%%%%%%%%%%%%%%%%%%%%%%%%%%%%%%%%%%%%%%%%%%%%%%%%%%%%%%%%%%%%%%%%%%%%%%
%%%%%%%%%%%%%%%%%%%%%%%%%%%%%%%%%%%%%%%%%%%%%%%%%%%%%%%%%%%%%%%%%%%%%%%%%%%%%%%
\section{Supplemental figures}


\begin{landscape}

\begin{figure}[t!]
	\begin{center}
		\includegraphics*[width=1.05\linewidth,center]{artwork/gwss_bf_gam_gw_ug_gam_gq_gam.png}
	\end{center}
	\caption[Genome-wide selection scans in Burkina Faso \textit{An. gambiae}]{
	%%
	\textbf{Genome-wide selection scans in Burkina Faso \textit{An. gambiae}.} 
	%
	Solid triangles show known insecticide resistance genes, empty triangles show novel loci. 
	%	
	$H12$ summarises the haplotype frequency spectrum in moving windows \citep{Garud2015}. 
	%
	$|IHS|$ = absolute value of the integrated haplotype score \citep{Voight2006}; each marker shows the maximum value in moving windows of 200 SNPs.
	%
	$XPEHH$ = cross-population extended haplotype homozygosity \citep{Sabeti2007}; each marker shows the maximum value in moving windows of 500 SNPs.
	%
	$PBS$ = normalized population branching statistic \citep{Crawford2017} in moving windows of 500 SNPs, scaled as defined in Methods.
	%%  
	} 
	\label{fig:gwss_bf_gam}
\end{figure}


\begin{figure}[t!]
	\begin{center}
		\includegraphics*[width=1.05\linewidth,center]{artwork/gwss_gn_gam_gw_ug_gam_gq_gam.png}
	\end{center}
	\caption[Genome-wide selection scans in Guinea \textit{An. gambiae}]{
	%%
	\textbf{Genome-wide selection scans in Guinea \textit{An. gambiae}.} 
	%
	See Supplemental Fig. \ref{fig:gwss_bf_gam} for figure legend.
	%%  
	} 
	\label{fig:gwss_gn_gam}
\end{figure}


\begin{figure}[t!]
	\begin{center}
		\includegraphics*[width=1.05\linewidth,center]{artwork/gwss_cm_sav_gam_gw_ug_gam_gq_gam.png}
	\end{center}
	\caption[Genome-wide selection scans in Cameroon \textit{An. gambiae}]{
	%%
	\textbf{Genome-wide selection scans in Cameroon \textit{An. gambiae}.} 
	%
	See Supplemental Fig. \ref{fig:gwss_bf_gam} for figure legend.
	%%  
	} 
	\label{fig:gwss_cm_sav_gam}
\end{figure}


\begin{figure}[t!]
	\begin{center}
		\includegraphics*[width=1.05\linewidth,center]{artwork/gwss_ug_gam_gw_bf_gam_gq_gam.png}
	\end{center}
	\caption[Genome-wide selection scans in Uganda \textit{An. gambiae}]{
	%%
	\textbf{Genome-wide selection scans in Uganda \textit{An. gambiae}.} 
	%
	See Supplemental Fig. \ref{fig:gwss_bf_gam} for figure legend.
	%%  
	} 
	\label{fig:gwss_ug_gam}
\end{figure}


\begin{figure}[t!]
	\begin{center}
		\includegraphics*[width=1.05\linewidth,center]{artwork/gwss_gh_gam_gw_ug_gam_gq_gam.png}
	\end{center}
	\caption[Genome-wide selection scans in Ghana \textit{An. gambiae}]{
	%%
	\textbf{Genome-wide selection scans in Ghana \textit{An. gambiae}.} 
	%
	See Supplemental Fig. \ref{fig:gwss_bf_gam} for figure legend.
	%%  
	} 
	\label{fig:gwss_gh_gam}
\end{figure}


\begin{figure}[t!]
	\begin{center}
		\includegraphics*[width=1.05\linewidth,center]{artwork/gwss_ga_gam_gw_ug_gam_gq_gam.png}
	\end{center}
	\caption[Genome-wide selection scans in Gabon \textit{An. gambiae}]{
	%%
	\textbf{Genome-wide selection scans in Gabon \textit{An. gambiae}.} 
	%
	See Supplemental Fig. \ref{fig:gwss_bf_gam} for figure legend.
	%%  
	} 
	\label{fig:gwss_ga_gam}
\end{figure}


\begin{figure}[t!]
	\begin{center}
		\includegraphics*[width=1.05\linewidth,center]{artwork/gwss_gq_gam_gw_ug_gam_bf_col.png}
	\end{center}
	\caption[Genome-wide selection scans in Bioko \textit{An. gambiae}]{
	%%
	\textbf{Genome-wide selection scans in Bioko \textit{An. gambiae}.} 
	%
	See Supplemental Fig. \ref{fig:gwss_bf_gam} for figure legend.
	%%  
	} 
	\label{fig:gwss_gq_gam}
\end{figure}


\begin{figure}[t!]
	\begin{center}
		\includegraphics*[width=1.05\linewidth,center]{artwork/gwss_fr_gam_gw_ug_gam_gq_gam.png}
	\end{center}
	\caption[Genome-wide selection scans in Mayotte \textit{An. gambiae}]{
	%%
	\textbf{Genome-wide selection scans in Mayotte \textit{An. gambiae}.} 
	%
	See Supplemental Fig. \ref{fig:gwss_bf_gam} for figure legend.
	%%  
	} 
	\label{fig:gwss_fr_gam}
\end{figure}


\begin{figure}[t!]
	\begin{center}
		\includegraphics*[width=1.05\linewidth,center]{artwork/gwss_bf_col_gw_ao_col_gq_gam.png}
	\end{center}
	\caption[Genome-wide selection scans in Burkina Faso \textit{An. coluzzii}]{
	%%
	\textbf{Genome-wide selection scans in Burkina Faso \textit{An. coluzzii}.} 
	%
	See Supplemental Fig. \ref{fig:gwss_bf_gam} for figure legend.
	%%  
	} 
	\label{fig:gwss_bf_col}
\end{figure}


\begin{figure}[t!]
	\begin{center}
		\includegraphics*[width=1.05\linewidth,center]{artwork/gwss_ci_col_gw_ao_col_gq_gam.png}
	\end{center}
	\caption[Genome-wide selection scans in C\^{o}te d'Ivoire \textit{An. coluzzii}]{
	%%
	\textbf{Genome-wide selection scans in C\^{o}te d'Ivoire \textit{An. coluzzii}.} 
	%
	See Supplemental Fig. \ref{fig:gwss_bf_gam} for figure legend.
	%%  
	} 
	\label{fig:gwss_ci_col}
\end{figure}


\begin{figure}[t!]
	\begin{center}
		\includegraphics*[width=1.05\linewidth,center]{artwork/gwss_gh_col_gw_ao_col_gq_gam.png}
	\end{center}
	\caption[Genome-wide selection scans in Ghana \textit{An. coluzzii}]{
	%%
	\textbf{Genome-wide selection scans in Ghana \textit{An. coluzzii}.} 
	%
	See Supplemental Fig. \ref{fig:gwss_bf_gam} for figure legend.
	%%  
	} 
	\label{fig:gwss_gh_col}
\end{figure}


\begin{figure}[t!]
	\begin{center}
		\includegraphics*[width=1.05\linewidth,center]{artwork/gwss_ao_col_gw_bf_col_gq_gam.png}
	\end{center}
	\caption[Genome-wide selection scans in Angola \textit{An. coluzzii}]{
	%%
	\textbf{Genome-wide selection scans in Angola \textit{An. coluzzii}.} 
	%
	See Supplemental Fig. \ref{fig:gwss_bf_gam} for figure legend.
	%%  
	} 
	\label{fig:gwss_ao_col}
\end{figure}


\begin{figure}[t!]
	\begin{center}
		\includegraphics*[width=1.05\linewidth,center]{artwork/gwss_gw_gq_gam_bf_col_gq_gam.png}
	\end{center}
	\caption[Genome-wide selection scans in Guinea-Bissau]{
	%%
	\textbf{Genome-wide selection scans in Guinea-Bissau.} 
	%
	See Supplemental Fig. \ref{fig:gwss_bf_gam} for figure legend.
	%%  
	} 
	\label{fig:gwss_gw}
\end{figure}



\begin{figure}[t!]
	\begin{center}
		\includegraphics*[width=1.05\linewidth,center]{artwork/gwss_gm_gq_gam_bf_col_gq_gam.png}
	\end{center}
	\caption[Genome-wide selection scans in The Gambia]{
	%%
	\textbf{Genome-wide selection scans in The Gambia.} 
	%
	See Supplemental Fig. \ref{fig:gwss_bf_gam} for figure legend.
	%%  
	} 
	\label{fig:gwss_gm}
\end{figure}


\end{landscape}


%%%%%%%%%%%%%%%%%%%%%%%%%%%%%%%%%%%%%%%%%%%%%%%%%%%%%%%%%%%%%%%%%%%%%%%%%%%%%%
\clearpage


\begin{figure}[t!]
	\begin{center}
		\includegraphics*[width=1\linewidth,center]{artwork/locus_gste2_h12_pdist.png}
	\end{center}
	\caption[\textit{Gste2} gene, $H12$ selection signals]{
	%
	\textbf{\textit{Gste2} gene, $H12$ selection signals.}
	%
	$H12_{peak}$ = maximum value of $H12$ found within 500 kbp of target gene. 
	%
	$pos(H12_{peak})$ = location of $H12_{peak}$ relative to target gene center.
	%	
	$P_{98}$ = 98th percentile of all values genome-wide.
	%
	Populations shown are those where 20 or more values above the 98th percentile were found within 200 kbp of the target gene.
	%  
	} 
	\label{fig:locus_gste2_h12}
\end{figure}


%%%%%%%%%%%%%%%%%%%%%%%%%%%%%%%%%%%%%%%%%%%%%%%%%%%%%%%%%%%%%%%%%%%%%%%%%%%%%%%
\clearpage


\begin{figure}[t!]
	\begin{center}
		\includegraphics*[width=1\linewidth,center]{artwork/locus_gste2_ihs_pdist.png}
	\end{center}
	\caption[\textit{Gste2} gene, $IHS$ selection signals]{
	%%
	\textbf{\textit{Gste2} gene, $IHS$ selection signals.}
	%
	Each marker shows the maximum $|IHS|$ value in moving windows of 50 SNPs. 
	%
	$|IHS|_{peak}$ = maximum value of $|IHS|$ found within 500 kbp of target gene. 
	%
	$pos(|IHS|_{peak})$ = location of $|IHS|_{peak}$ relative to target gene centre.
	%
	$P_{98}$ = 98th percentile of all values genome-wide.
	%
	Populations shown are those where 20 or more values above the 98th percentile were found within 200 kbp of the target gene.
	%%  
	} 
	\label{fig:locus_gste2_ihs}
\end{figure}


%%%%%%%%%%%%%%%%%%%%%%%%%%%%%%%%%%%%%%%%%%%%%%%%%%%%%%%%%%%%%%%%%%%%%%%%%%%%%%%
\clearpage


\begin{figure}[t!]
	\begin{center}
		\includegraphics*[width=1\linewidth,center]{artwork/locus_gste2_xpehh_pdist.png}
	\end{center}
	\caption[\textit{Gste2} gene, $XPEHH$ selection signals]{
	%%
	\textbf{\textit{Gste2} gene, $XPEHH$ selection signals.}
	%
	Each marker shows the maximum $XPEHH$ value in moving windows of 100 SNPs. 
	%
	$XPEHH_{peak}$ = maximum value of $XPEHH$ found within 500 kbp of target gene. 
	%
	$pos(XPEHH_{peak})$ = location of $XPEHH_{peak}$ relative to target gene centre.
	%
	$P_{98}$ = 98th percentile of all values genome-wide.
	%
	Populations shown are those where 20 or more values above the 98th percentile were found within 200 kbp of the target gene.
	%%  
	} 
	\label{fig:locus_gste2_xpehh}
\end{figure}


%%%%%%%%%%%%%%%%%%%%%%%%%%%%%%%%%%%%%%%%%%%%%%%%%%%%%%%%%%%%%%%%%%%%%%%%%%%%%%%
\clearpage
\begin{figure}[t!]
	\begin{center}
		\includegraphics*[width=1\linewidth,center]{artwork/locus_gste2_pbs_pdist.png}
	\end{center}
	\caption[\textit{Gste2} gene, $PBS$ selection signals]{
	%%
	\textbf{\textit{Gste2} gene, $PBS$ selection signals.} 
	%
	$PBS_{peak}$ = maximum value of $PBS$ found within 500 kbp of target gene. 
	%
	$pos(PBS_{peak})$ = location of $PBS_{peak}$ relative to target gene centre.
	%
	$P_{98}$ = 98th percentile of all values genome-wide.
	%
	Populations shown are those where 20 or more values above the 98th percentile were found within 200 kbp of the target gene.
	%%  
	} 
	\label{fig:locus_gste2_pbs}
\end{figure}


%%%%%%%%%%%%%%%%%%%%%%%%%%%%%%%%%%%%%%%%%%%%%%%%%%%%%%%%%%%%%%%%%%%%%%%%%%%%%%%
\clearpage


\begin{figure}[t!]
	\begin{center}
		\includegraphics*[width=1\linewidth,center]{artwork/locus_cyp6p3_h12_pdist.png}
	\end{center}
	\caption[\textit{Cyp6p3} gene, $H12$ selection signals]{
	%%
	\textbf{\textit{Cyp6p3} gene, $H12$ selection signals.}
	%
	See Supplemental Fig. \ref{fig:locus_gste2_h12} for legend.
	%%  
	} 
	\label{fig:locus_cyp6p3_h12}
\end{figure}


%%%%%%%%%%%%%%%%%%%%%%%%%%%%%%%%%%%%%%%%%%%%%%%%%%%%%%%%%%%%%%%%%%%%%%%%%%%%%%%
\clearpage


\begin{figure}[t!]
	\begin{center}
		\includegraphics*[width=1\linewidth,center]{artwork/locus_cyp6p3_ihs_pdist.png}
	\end{center}
	\caption[\textit{Cyp6p3} gene, $IHS$ selection signals]{
	%%
	\textbf{\textit{Cyp6p3} gene, $IHS$ selection signals.}
	%
	See Supplemental Fig. \ref{fig:locus_gste2_ihs} for legend.
	%%  
	} 
	\label{fig:locus_cyp6p3_ihs}
\end{figure}


%%%%%%%%%%%%%%%%%%%%%%%%%%%%%%%%%%%%%%%%%%%%%%%%%%%%%%%%%%%%%%%%%%%%%%%%%%%%%%%
\clearpage


\begin{figure}[t!]
	\begin{center}
		\includegraphics*[width=1\linewidth,center]{artwork/locus_cyp6p3_xpehh_pdist.png}
	\end{center}
	\caption[\textit{Cyp6p3} gene, $XPEHH$ selection signals]{
	%%
	\textbf{\textit{Cyp6p3} gene, $XPEHH$ selection signals.}
	%
	See Supplemental Fig. \ref{fig:locus_gste2_xpehh} for legend.
	%%  
	} 
	\label{fig:locus_cyp6p3_xpehh}
\end{figure}


%%%%%%%%%%%%%%%%%%%%%%%%%%%%%%%%%%%%%%%%%%%%%%%%%%%%%%%%%%%%%%%%%%%%%%%%%%%%%%%
\clearpage
\begin{figure}[t!]
	\begin{center}
		\includegraphics*[width=1\linewidth,center]{artwork/locus_cyp6p3_pbs_pdist.png}
	\end{center}
	\caption[\textit{Cyp6p3} gene, $PBS$ selection signals]{
	%%
	\textbf{\textit{Cyp6p3} gene, $PBS$ selection signals.} 
	%
	See Supplemental Fig. \ref{fig:locus_gste2_pbs} for legend.
	%%  
	} 
	\label{fig:locus_cyp6p3_pbs}
\end{figure}


%%%%%%%%%%%%%%%%%%%%%%%%%%%%%%%%%%%%%%%%%%%%%%%%%%%%%%%%%%%%%%%%%%%%%%%%%%%%%%%
\clearpage


\begin{figure}[t!]
	\begin{center}
		\includegraphics*[width=1\linewidth,center]{artwork/locus_cyp9k1_h12_pdist.png}
	\end{center}
	\caption[\textit{Cyp9k1} gene, $H12$ selection signals]{
	%%
	\textbf{\textit{Cyp9k1} gene, $H12$ selection signals.}
	%
	See Supplemental Fig. \ref{fig:locus_gste2_h12} for legend.
	%%  
	} 
	\label{fig:locus_cyp9k1_h12}
\end{figure}


%%%%%%%%%%%%%%%%%%%%%%%%%%%%%%%%%%%%%%%%%%%%%%%%%%%%%%%%%%%%%%%%%%%%%%%%%%%%%%%
\clearpage


\begin{figure}[t!]
	\begin{center}
		\includegraphics*[width=1\linewidth,center]{artwork/locus_cyp9k1_ihs_pdist.png}
	\end{center}
	\caption[\textit{Cyp9k1} gene, $IHS$ selection signals]{
	%%
	\textbf{\textit{Cyp9k1} gene, $IHS$ selection signals.}
	%
	See Supplemental Fig. \ref{fig:locus_gste2_ihs} for legend.
	%%  
	} 
	\label{fig:locus_cyp9k1_ihs}
\end{figure}


%%%%%%%%%%%%%%%%%%%%%%%%%%%%%%%%%%%%%%%%%%%%%%%%%%%%%%%%%%%%%%%%%%%%%%%%%%%%%%%
\clearpage


\begin{figure}[t!]
	\begin{center}
		\includegraphics*[width=1\linewidth,center]{artwork/locus_cyp9k1_xpehh_pdist.png}
	\end{center}
	\caption[\textit{Cyp9k1} gene, $XPEHH$ selection signals]{
	%%
	\textbf{\textit{Cyp9k1} gene, $XPEHH$ selection signals.}
	%
	See Supplemental Fig. \ref{fig:locus_gste2_xpehh} for legend.
	%%  
	} 
	\label{fig:locus_cyp9k1_xpehh}
\end{figure}


%%%%%%%%%%%%%%%%%%%%%%%%%%%%%%%%%%%%%%%%%%%%%%%%%%%%%%%%%%%%%%%%%%%%%%%%%%%%%%%
\clearpage


\begin{figure}[t!]
	\begin{center}
		\includegraphics*[width=1\linewidth,center]{artwork/locus_cyp9k1_pbs_pdist.png}
	\end{center}
	\caption[\textit{Cyp9k1} gene, $PBS$ selection signals]{
	%%
	\textbf{\textit{Cyp9k1} gene, $PBS$ selection signals.}
	%
	See Supplemental Fig. \ref{fig:locus_gste2_pbs} for legend.
	%%  
	} 
	\label{fig:locus_cyp9k1_pbs}
\end{figure}


%%%%%%%%%%%%%%%%%%%%%%%%%%%%%%%%%%%%%%%%%%%%%%%%%%%%%%%%%%%%%%%%%%%%%%%%%%%%%%%
\clearpage


\begin{figure}[t!]
	\begin{center}
		\includegraphics*[width=1\linewidth,center]{artwork/locus_vgsc_h12_pdist.png}
	\end{center}
	\caption[\textit{Vgsc} gene, $H12$ selection signals, physical distance]{
	%%
	\textbf{\textit{Vgsc} gene, $H12$ selection signals, physical distance.}
	%
	See Supplemental Fig. \ref{fig:locus_gste2_h12} for legend.
	%%  
	} 
	\label{fig:locus_vgsc_h12}
\end{figure}


%%%%%%%%%%%%%%%%%%%%%%%%%%%%%%%%%%%%%%%%%%%%%%%%%%%%%%%%%%%%%%%%%%%%%%%%%%%%%%%
\clearpage


\begin{figure}[t!]
	\begin{center}
		\includegraphics*[width=1\linewidth,center]{artwork/locus_vgsc_h12_gdist.png}
	\end{center}
	\caption[\textit{Vgsc} gene, $H12$ selection signals, genetic distance]{
	%%
	\textbf{\textit{Vgsc} gene, $H12$ selection signals, genetic distance.}
	%
	Genetic distance was computed assuming a recombination rate of 2.0 cM/Mbp in euchromatic regions and 0.5 cM/Mbp in heterochromatic regions.  
	%
	See Supplemental Fig. \ref{fig:locus_gste2_h12} for legend.
	%%  
	} 
	\label{fig:locus_vgsc_h12_gdist}
\end{figure}


%%%%%%%%%%%%%%%%%%%%%%%%%%%%%%%%%%%%%%%%%%%%%%%%%%%%%%%%%%%%%%%%%%%%%%%%%%%%%%%
\clearpage


\begin{figure}[t!]
	\begin{center}
		\includegraphics*[width=1\linewidth,center]{artwork/locus_vgsc_pbs_gdist.png}
	\end{center}
	\caption[\textit{Vgsc} gene, $PBS$ selection signals, genetic distance]{
	%%
	\textbf{\textit{Vgsc} gene, $PBS$ selection signals, genetic distance.}
	%
	See Supplemental Fig. \ref{fig:locus_gste2_pbs} for legend.
	%%  
	} 
	\label{fig:locus_vgsc_pbs_gdist}
\end{figure}


%%%%%%%%%%%%%%%%%%%%%%%%%%%%%%%%%%%%%%%%%%%%%%%%%%%%%%%%%%%%%%%%%%%%%%%%%%%%%%%
\clearpage


\begin{figure}[t!]
	\begin{center}
		\includegraphics*[width=1\linewidth,center]{artwork/locus_gaba_h12_pdist.png}
	\end{center}
	\caption[\textit{Gaba} gene, $H12$ selection signals]{
	%%
	\textbf{\textit{Gaba} gene, $H12$ selection signals.}
	%
	See Supplemental Fig. \ref{fig:locus_gste2_h12} for legend.
	%%  
	} 
	\label{fig:locus_gaba_h12}
\end{figure}


%%%%%%%%%%%%%%%%%%%%%%%%%%%%%%%%%%%%%%%%%%%%%%%%%%%%%%%%%%%%%%%%%%%%%%%%%%%%%%%
\clearpage


\begin{figure}[t!]
	\begin{center}
		\includegraphics*[width=1\linewidth,center]{artwork/locus_gaba_ihs_pdist.png}
	\end{center}
	\caption[\textit{Gaba} gene, $IHS$ selection signals]{
	%%
	\textbf{\textit{Gaba} gene, $IHS$ selection signals.}
	%
	See Supplemental Fig. \ref{fig:locus_gste2_ihs} for legend.
	%%  
	} 
	\label{fig:locus_gaba_ihs}
\end{figure}


%%%%%%%%%%%%%%%%%%%%%%%%%%%%%%%%%%%%%%%%%%%%%%%%%%%%%%%%%%%%%%%%%%%%%%%%%%%%%%%
\clearpage


\begin{figure}[t!]
	\begin{center}
		\includegraphics*[width=1\linewidth,center]{artwork/locus_gaba_xpehh_pdist.png}
	\end{center}
	\caption[\textit{Gaba} gene, $XPEHH$ selection signals]{
	%%
	\textbf{\textit{Gaba} gene, $XPEHH$ selection signals.}
	%
	See Supplemental Fig. \ref{fig:locus_gste2_xpehh} for legend.
	%%  
	} 
	\label{fig:locus_gaba_xpehh}
\end{figure}


%%%%%%%%%%%%%%%%%%%%%%%%%%%%%%%%%%%%%%%%%%%%%%%%%%%%%%%%%%%%%%%%%%%%%%%%%%%%%%%
\clearpage


\begin{figure}[t!]
	\begin{center}
		\includegraphics*[width=1\linewidth,center]{artwork/locus_ace1_h12_pdist.png}
	\end{center}
	\caption[\textit{Ace1} gene, $H12$ selection signals]{
	%%
	\textbf{\textit{Ace1} gene, $H12$ selection signals.}
	%
	See Supplemental Fig. \ref{fig:locus_gste2_h12} for legend.
	%%  
	} 
	\label{fig:locus_ace1_h12}
\end{figure}


%%%%%%%%%%%%%%%%%%%%%%%%%%%%%%%%%%%%%%%%%%%%%%%%%%%%%%%%%%%%%%%%%%%%%%%%%%%%%%%
\clearpage


\begin{figure}[t!]
	\begin{center}
		\includegraphics*[width=1\linewidth,center]{artwork/locus_ace1_ihs_pdist.png}
	\end{center}
	\caption[\textit{Ace1} gene, $IHS$ selection signals]{
	%%
	\textbf{\textit{Ace1} gene, $IHS$ selection signals.}
	%
	See Supplemental Fig. \ref{fig:locus_gste2_ihs} for legend.
	%%  
	} 
	\label{fig:locus_ace1_ihs}
\end{figure}


%%%%%%%%%%%%%%%%%%%%%%%%%%%%%%%%%%%%%%%%%%%%%%%%%%%%%%%%%%%%%%%%%%%%%%%%%%%%%%%
\clearpage


\begin{figure}[t!]
	\begin{center}
		\includegraphics*[width=1\linewidth,center]{artwork/locus_ace1_xpehh_pdist.png}
	\end{center}
	\caption[\textit{Ace1} gene, $XPEHH$ selection signals]{
	%%
	\textbf{\textit{Ace1} gene, $XPEHH$ selection signals.}
	%
	See Supplemental Fig. \ref{fig:locus_gste2_xpehh} for legend.
	%%  
	} 
	\label{fig:locus_ace1_xpehh}
\end{figure}


%%%%%%%%%%%%%%%%%%%%%%%%%%%%%%%%%%%%%%%%%%%%%%%%%%%%%%%%%%%%%%%%%%%%%%%%%%%%%%%
\clearpage


\begin{figure}[t!]
	\begin{center}
		\includegraphics*[width=1\linewidth,center]{artwork/locus_ace1_pbs_pdist.png}
	\end{center}
	\caption[\textit{Ace1} gene, $PBS$ selection signals]{
	%%
	\textbf{\textit{Ace1} gene, $PBS$ selection signals.}
	%
	See Supplemental Fig. \ref{fig:locus_gste2_pbs} for legend.
	%%  
	} 
	\label{fig:locus_ace1_pbs}
\end{figure}



%%%%%%%%%%%%%%%%%%%%%%%%%%%%%%%%%%%%%%%%%%%%%%%%%%%%%%%%%%%%%%%%%%%%%%%%%%%%%%%
\clearpage


\begin{figure}[t!]
	\begin{center}
		\includegraphics*[width=1\linewidth,center]{artwork/exponential_fit_examples.png}
	\end{center}
	\caption[$H12$ exponential peak fitting examples]{
	%%
	\textbf{$H12$ exponential peak fitting examples.}
	%
	Each panel shows an exponential peak model fitted to $H12$ values via nonlinear least squares regression, at either the \textit{Gste2} or \textit{Cyp6p3} locus.
	%
	$AIC$ = Akaike Information Criterion.
	%
	$BIC$ = Bayesian Information Criterion.
	%
	$\chi^{2}$ = residual sum of squares.
	%
	$\Delta_{i}$ = AIC difference between exponential peak model and constant (null) model.
	%
	The variables annotated were fitted via least squares regression; shown as \textit{value} $\pm$ \textit{standard error} (\textit{relative error}).
	%%  
	} 
	\label{fig:exponential_fit_examples}
\end{figure}



%%%%%%%%%%%%%%%%%%%%%%%%%%%%%%%%%%%%%%%%%%%%%%%%%%%%%%%%%%%%%%%%%%%%%%%%%%%%%%%
\clearpage

%%%%%%%%%%%%%%%%%%%%%%%%%%%%%%%%%%%%%%%%%%%%%%%%%%%%%%%%%%%%%%%%%%%%%%%%%%%%%%%
%%%%%%%%%%%%%%%%%%%%%%%%%%%%%%%%%%%%%%%%%%%%%%%%%%%%%%%%%%%%%%%%%%%%%%%%%%%%%%%
\section{Supplemental tables}


%%%%%%%%%%%%%%%%%%%%%%%%%%%%%%%%%%%%%%%%%%%%%%%%%%%%%%%%%%%%%%%%%%%%%%%%%%%%%%%
\clearpage


\begin{table}
\centering
\begin{threeparttable}

\caption[$H12$ selection signals at known insecticide-resistance genes]{
%
\textbf{$H12$ selection signals at known insecticide-resistance genes.}
%
Values shown are peak value of $H12$ found within 500 kbp of the target gene.
%
Peak values are only shown for populations where more than 20 values above the 98th percentile of all values genome-wide were found within 200 kbp of the target gene.
%
}

\label{tbl:locus_peaks_h12}

\begin{tabular}{lllllll}
\toprule
                              Population & Ace1 & Cyp6p3 & Cyp9k1 & Gaba & Gste2 & Vgsc \\
\midrule
            Angola \textit{An. coluzzii} &    - &      - &      - &    - &     - & 0.39 \\
      Burkina Faso \textit{An. coluzzii} &    - &   0.36 &   0.92 & 0.32 &  0.25 & 0.81 \\
       Burkina Faso \textit{An. gambiae} &    - &   0.36 &   0.38 & 0.15 &  0.38 & 0.96 \\
     Cote d'Ivoire \textit{An. coluzzii} & 0.21 &   0.32 &   0.68 & 0.36 &  0.33 & 0.85 \\
 Cameroon (savanna) \textit{An. gambiae} &    - &   0.27 &      - &    - &  0.40 & 0.20 \\
            Mayotte \textit{An. gambiae} &    - &      - &      - &    - &     - &    - \\
              Gabon \textit{An. gambiae} &    - &      - &   0.32 & 0.36 &  0.14 & 0.19 \\
             Ghana \textit{An. coluzzii} &    - &   0.23 &   0.45 & 0.37 &  0.48 & 0.88 \\
              Ghana \textit{An. gambiae} & 0.41 &   0.51 &      - & 0.92 &     - & 1.00 \\
                              The Gambia &    - &   0.44 &   0.25 &    - &     - &    - \\
             Guinea \textit{An. gambiae} &    - &   0.30 &   0.22 & 0.22 &  0.32 & 0.98 \\
              Bioko \textit{An. gambiae} &    - &      - &   0.23 & 0.54 &     - &    - \\
                           Guinea-Bissau &    - &      - &      - &    - &     - &    - \\
                                   Kenya &    - &      - &      - &    - &     - &    - \\
             Uganda \textit{An. gambiae} &    - &   0.71 &      - &    - &  0.31 & 0.96 \\
\bottomrule
\end{tabular}


%\begin{tablenotes}
%\item[1] @@TODO
%\end{tablenotes}

\end{threeparttable}
\end{table}


\begin{table}
\centering
\begin{threeparttable}

\caption[Summary statistics for $H12$ selection signals at known insecticide-resistance genes]{
%
\textbf{Summary statistics for $H12$ selection signals at known insecticide-resistance genes.}
%
No. populations = number of populations included in the analysis, where more than 20 values above the 98th percentile of all values genome-wide were found within 200 kbp of the target gene.
%
$H12_{peak}$ = maximum value of $H12$ found within 500 kbp of the target gene.
%
$pos(H12_{peak})$ = position of peak value, relative to gene.
%
MAE = mean absolute error, i.e., mean distance from peak value to gene.
%
}

\label{tbl:locus_stats_h12}

\begin{tabular}{lrrrrrr}
\toprule
           Locus &   Populations & \multicolumn{2}{l}{$H12_{peak}$} & \multicolumn{3}{l}{$pos(H12_{peak})$ (kbp)} \\
                 & No. with peak &          Min &  Max &                     Min &    Max &   MAE \\
\midrule
  \textit{Gste2} &             7 &         0.25 & 0.48 &                   -79.2 &  +13.9 &  20.5 \\
 \textit{Cyp6p3} &             9 &         0.23 & 0.71 &                   -43.2 &  +49.2 &  32.8 \\
   \textit{Vgsc} &             8 &         0.39 & 1.00 &                  -341.9 & +245.2 & 137.9 \\
   \textit{Gaba} &             8 &         0.15 & 0.92 &                   +25.6 & +273.1 &  91.8 \\
   \textit{Ace1} &             2 &         0.21 & 0.41 &                  -371.4 & +159.2 & 265.3 \\
 \textit{Cyp9k1} &             8 &         0.22 & 0.92 &                  -467.4 & +282.5 & 180.9 \\
\bottomrule
\end{tabular}


%\begin{tablenotes}
%\item[1] @@TODO
%\end{tablenotes}

\end{threeparttable}
\end{table}


%%%%%%%%%%%%%%%%%%%%%%%%%%%%%%%%%%%%%%%%%%%%%%%%%%%%%%%%%%%%%%%%%%%%%%%%%%%%%%%
\clearpage


\begin{table}
\centering
\begin{threeparttable}

\caption[$IHS$ selection signals at known insecticide-resistance genes]{
%
\textbf{$IHS$ selection signals at known insecticide-resistance genes.}
%
Values shown are peak absolute value of $IHS$ found in proximity to the gene.
%
Peak values are only shown for populations where more than 20 values above the 98th percentile of all values genome-wide were found within 200 kbp of the target gene.
%
}

\label{tbl:locus_peaks_ihs}

\begin{tabular}{lllllll}
\toprule
                              Population & \textit{Gste2} & \textit{Cyp6p3} & \textit{Cyp9k1} & \textit{Vgsc} & \textit{Gaba} & \textit{Ace1} \\
\midrule
            Angola \textit{An. coluzzii} &              - &            7.22 &               - &             - &             - &             - \\
      Burkina Faso \textit{An. coluzzii} &           9.63 &            6.61 &               - &             - &          8.11 &             - \\
       Burkina Faso \textit{An. gambiae} &           9.33 &               - &            5.86 &             - &          6.86 &             - \\
     Cote d'Ivoire \textit{An. coluzzii} &           5.86 &            4.80 &               - &             - &          5.68 &          5.90 \\
 Cameroon (savanna) \textit{An. gambiae} &           9.00 &            5.99 &            7.03 &             - &             - &             - \\
            Mayotte \textit{An. gambiae} &              - &               - &               - &             - &             - &             - \\
              Gabon \textit{An. gambiae} &              - &               - &               - &             - &             - &             - \\
             Ghana \textit{An. coluzzii} &           6.56 &               - &            5.64 &             - &          6.50 &             - \\
              Ghana \textit{An. gambiae} &           5.42 &               - &               - &             - &             - &          4.25 \\
                              The Gambia &              - &            6.97 &            5.14 &             - &             - &             - \\
             Guinea \textit{An. gambiae} &           8.75 &            5.58 &            7.27 &             - &          7.51 &             - \\
              Bioko \textit{An. gambiae} &              - &               - &               - &             - &             - &             - \\
                           Guinea-Bissau &              - &               - &            4.91 &             - &             - &             - \\
                                   Kenya &              - &               - &               - &             - &             - &             - \\
             Uganda \textit{An. gambiae} &           7.23 &            5.87 &            6.68 &             - &             - &             - \\
\bottomrule
\end{tabular}


%\begin{tablenotes}
%\item[1] @@TODO
%\end{tablenotes}

\end{threeparttable}
\end{table}


\begin{table}
\centering
\begin{threeparttable}

\caption[Summary statistics for $IHS$ selection signals at known insecticide-resistance genes]{
%
\textbf{Summary statistics for $IHS$ selection signals at known insecticide-resistance genes.}
%
No. populations = number of populations included in the analysis, where more than 20 values above the 98th percentile of all values genome-wide were found within 200 kbp of the target gene.
%
$|IHS|_{peak}$ = maximum absolute value of $IHS$ found within 500 kbp of the target gene.
%
$pos(|IHS|_{peak})$ = position of peak value, relative to gene.
%
MAE = mean absolute error, i.e., mean distance from peak value to gene.
%
}

\label{tbl:locus_stats_ihs}

\begin{tabular}{lrrrrrr}
\toprule
           Locus & \multicolumn{3}{l}{$|IHS|_{peak}$} & \multicolumn{3}{l}{$pos(|IHS|_{peak})$ (kbp)} \\
                 & No. populations &  Min &  Max &                       Min &    Max &   MAE \\
\midrule
  \textit{Gste2} &               8 & 5.42 & 9.63 &                    -259.6 &   +4.3 &  97.4 \\
 \textit{Cyp6p3} &               7 & 4.80 & 7.22 &                    -605.3 & +277.4 & 226.6 \\
 \textit{Cyp9k1} &               1 & 7.03 & 7.03 &                    +277.6 & +277.6 & 277.6 \\
   \textit{Vgsc} &               0 &    - &    - &                         - &      - &     - \\
   \textit{Gaba} &               5 & 5.68 & 8.11 &                     +52.1 & +142.9 &  99.9 \\
   \textit{Ace1} &               2 & 5.00 & 5.90 &                    -509.2 & +228.3 & 368.8 \\
\bottomrule
\end{tabular}


%\begin{tablenotes}
%\item[1] @@TODO
%\end{tablenotes}

\end{threeparttable}
\end{table}


%%%%%%%%%%%%%%%%%%%%%%%%%%%%%%%%%%%%%%%%%%%%%%%%%%%%%%%%%%%%%%%%%%%%%%%%%%%%%%%
\clearpage


\begin{table}
\centering
\begin{threeparttable}

\caption[$XPEHH$ selection signals at known insecticide-resistance genes]{
%
\textbf{$XPEHH$ selection signals at known insecticide-resistance genes.}
%
Values shown are peak value of $XPEHH$ found in proximity to the gene.
%
Peak values are only shown for populations where more than 20 values above the 98th percentile of all values genome-wide were found within 200 kbp of the target gene.
%
}

\label{tbl:locus_peaks_xpehh}

\begin{tabular}{lllllll}
\toprule
                              Population & \textit{Gste2} & \textit{Cyp6p3} & \textit{Cyp9k1} & \textit{Vgsc} & \textit{Gaba} & \textit{Ace1} \\
\midrule
            Angola \textit{An. coluzzii} &              - &               - &               - &             - &             - &             - \\
      Burkina Faso \textit{An. coluzzii} &           5.40 &            4.00 &            5.54 &             - &          4.69 &             - \\
       Burkina Faso \textit{An. gambiae} &           5.33 &            4.43 &            3.68 &             - &             - &             - \\
     Cote d'Ivoire \textit{An. coluzzii} &           4.81 &            4.22 &            5.09 &             - &          5.56 &             - \\
 Cameroon (savanna) \textit{An. gambiae} &           5.95 &            3.55 &               - &             - &             - &             - \\
            Mayotte \textit{An. gambiae} &              - &               - &               - &             - &             - &             - \\
              Gabon \textit{An. gambiae} &           4.33 &               - &               - &             - &             - &             - \\
             Ghana \textit{An. coluzzii} &           5.39 &               - &            5.10 &             - &          5.68 &             - \\
              Ghana \textit{An. gambiae} &              - &               - &               - &             - &             - &          5.02 \\
                              The Gambia &              - &            5.47 &               - &             - &             - &             - \\
             Guinea \textit{An. gambiae} &           4.72 &            3.91 &            2.68 &             - &          4.71 &             - \\
              Bioko \textit{An. gambiae} &              - &               - &               - &             - &          5.95 &             - \\
                           Guinea-Bissau &              - &               - &               - &             - &             - &             - \\
                                   Kenya &              - &               - &               - &             - &             - &             - \\
             Uganda \textit{An. gambiae} &           4.91 &            5.62 &               - &             - &             - &             - \\
\bottomrule
\end{tabular}


%\begin{tablenotes}
%\item[1] @@TODO
%\end{tablenotes}

\end{threeparttable}
\end{table}


\begin{table}
\centering
\begin{threeparttable}

\caption[Summary statistics for $XPEHH$ selection signals at known insecticide-resistance genes]{
%
\textbf{Summary statistics for $XPEHH$ selection signals at known insecticide-resistance genes.}
%
No. populations = number of populations included in the analysis, where more than 20 values above the 98th percentile of all values genome-wide were found within 200 kbp of the target gene.
%
$XPEHH_{peak}$ = maximum value of XPEHH found within 500 kbp of the target gene.
%
$pos(XPEHH_{peak})$ = position of peak value, relative to gene.
%
MAE = mean absolute error, i.e., mean distance from peak value to gene.
%
}

\label{tbl:locus_stats_xpehh}

\begin{tabular}{lrrrrrr}
\toprule
           Locus & \multicolumn{3}{l}{$XPEHH_{peak}$} & \multicolumn{3}{l}{$pos(XPEHH_{peak})$ (kbp)} \\
                 & No. populations &  Min &  Max &                       Min &    Max &   MAE \\
\midrule
  \textit{Gste2} &               8 & 4.33 & 5.95 &                     -93.8 &   +7.1 &  17.0 \\
 \textit{Cyp6p3} &               7 & 3.55 & 5.62 &                     -65.6 &  +32.9 &  43.9 \\
 \textit{Cyp9k1} &               5 & 2.68 & 5.54 &                    +108.2 & +331.7 & 159.8 \\
   \textit{Vgsc} &               0 &    - &    - &                         - &      - &     - \\
   \textit{Gaba} &               5 & 4.69 & 5.95 &                     +54.6 & +114.6 &  86.8 \\
   \textit{Ace1} &               1 & 5.02 & 5.02 &                    -108.1 & -108.1 & 108.1 \\
\bottomrule
\end{tabular}


%\begin{tablenotes}
%\item[1] @@TODO
%\end{tablenotes}

\end{threeparttable}
\end{table}


%%%%%%%%%%%%%%%%%%%%%%%%%%%%%%%%%%%%%%%%%%%%%%%%%%%%%%%%%%%%%%%%%%%%%%%%%%%%%%%
\clearpage


\begin{table}
\centering
\begin{threeparttable}

\caption[$PBS$ selection signals at known insecticide-resistance genes]{
%
\textbf{$PBS$ selection signals at known insecticide-resistance genes.}
%
Values shown are peak value of $PBS$ found in proximity to the gene, scaled as defined in Methods.
%
Peak values are only shown for populations where more than 20 values above the 98th percentile of all values genome-wide were found within 200 kbp of the target gene.
%
}

\label{tbl:locus_peaks_pbs}

\begin{tabular}{lllllll}
\toprule
                              Population & \textit{Gste2} & \textit{Cyp6p3} & \textit{Cyp9k1} & \textit{Vgsc} & \textit{Gaba} & \textit{Ace1} \\
\midrule
            Angola \textit{An. coluzzii} &              - &               - &               - &             - &             - &             - \\
      Burkina Faso \textit{An. coluzzii} &          11.52 &               - &           18.60 &             - &             - &             - \\
       Burkina Faso \textit{An. gambiae} &          37.47 &           15.69 &           27.55 &         55.66 &             - &             - \\
     Cote d'Ivoire \textit{An. coluzzii} &           6.63 &            4.68 &            7.14 &             - &             - &             - \\
 Cameroon (savanna) \textit{An. gambiae} &          35.65 &           12.52 &               - &         17.28 &             - &             - \\
            Mayotte \textit{An. gambiae} &              - &               - &               - &             - &             - &             - \\
              Gabon \textit{An. gambiae} &              - &               - &               - &          7.19 &             - &             - \\
             Ghana \textit{An. coluzzii} &           8.23 &               - &            8.92 &             - &             - &             - \\
              Ghana \textit{An. gambiae} &          19.71 &            6.62 &               - &         26.08 &             - &         11.76 \\
                              The Gambia &              - &            9.63 &               - &             - &             - &             - \\
             Guinea \textit{An. gambiae} &          31.67 &           11.86 &           17.68 &         55.74 &             - &             - \\
              Bioko \textit{An. gambiae} &              - &               - &               - &             - &             - &             - \\
                           Guinea-Bissau &              - &               - &               - &             - &             - &             - \\
                                   Kenya &              - &               - &               - &             - &             - &             - \\
             Uganda \textit{An. gambiae} &          24.09 &           13.85 &               - &         25.18 &             - &             - \\
\bottomrule
\end{tabular}


%\begin{tablenotes}
%\item[1] @@TODO
%\end{tablenotes}

\end{threeparttable}
\end{table}


\begin{table}
\centering
\begin{threeparttable}

\caption[Summary statistics for $PBS$ selection signals at known insecticide-resistance genes]{
%
\textbf{Summary statistics for $PBS$ selection signals at known insecticide-resistance genes.}
%
No. populations = number of populations included in the analysis, where more than 20 values above the 98th percentile of all values genome-wide were found within 200 kbp of the target gene.
%
$PBS_{peak}$ = maximum value of PBS found within 500 kbp of the target gene.
%
$pos(PBS_{peak})$ = position of peak value, relative to gene.
%
MAE = mean absolute error, i.e., mean distance from peak value to gene.
%
}

\label{tbl:locus_stats_pbs}

\begin{tabular}{lrrrrrr}
\toprule
           Locus & \multicolumn{3}{l}{$PBS_{peak}$} & \multicolumn{3}{l}{$pos(PBS_{peak})$ (kbp)} \\
                 & No. populations &   Min &   Max &                     Min &    Max &   MAE \\
\midrule
  \textit{Gste2} &               8 &  6.63 & 37.47 &                  -375.4 & +234.3 & 170.6 \\
 \textit{Cyp6p3} &               7 &  4.68 & 15.69 &                  -470.3 &  +46.0 & 109.4 \\
 \textit{Cyp9k1} &               5 &  7.14 & 27.55 &                  -297.5 & +152.8 & 128.9 \\
   \textit{Vgsc} &               6 &  7.19 & 55.74 &                  -336.2 & +253.7 & 256.4 \\
   \textit{Gaba} &               0 &     - &     - &                       - &      - &     - \\
   \textit{Ace1} &               1 & 11.76 & 11.76 &                    +5.7 &   +5.7 &   5.7 \\
\bottomrule
\end{tabular}


%\begin{tablenotes}
%\item[1] @@TODO
%\end{tablenotes}

\end{threeparttable}
\end{table}


%%%%%%%%%%%%%%%%%%%%%%%%%%%%%%%%%%%%%%%%%%%%%%%%%%%%%%%%%%%%%%%%%%%%%%%%%%%%%%%
\clearpage


%%%%%%%%%%%%%%%%%%%%%%%%%%%%%%%%%%%%%%%%%%%%%%%%%%%%%%%%%%%%%%%%%%%%%%%%%%%%%%%
%%%%%%%%%%%%%%%%%%%%%%%%%%%%%%%%%%%%%%%%%%%%%%%%%%%%%%%%%%%%%%%%%%%%%%%%%%%%%%%
\section{Supplemental methods}

@@TODO


\printbibliography


\end{document}
